\documentclass[journal]{IEEEtran}
\usepackage[a5paper, margin=10mm, onecolumn]{geometry}
\usepackage{tfrupee}

\setlength{\headheight}{1cm}
\setlength{\headsep}{0mm}

\usepackage{cite}
\usepackage{amsmath,amssymb,amsfonts,amsthm}
\usepackage{algorithmic}
\usepackage{graphicx}
\usepackage{textcomp}
\usepackage{xcolor}
\usepackage{listings}
\usepackage{enumitem}
\usepackage{mathtools}
\usepackage{gensymb}
\usepackage{comment}
\usepackage[breaklinks=true]{hyperref}
\usepackage{tkz-euclide}
\usepackage{longtable}
\usepackage{multirow}
\usepackage{hhline}
\usepackage{array}

\begin{document}

\bibliographystyle{IEEEtran}
\vspace{3cm}

\title{4.2.1.1}
\author{EE24BTECH11004 - Ankit Jainar}
\maketitle

\renewcommand{\thefigure}{\theenumi}
\renewcommand{\thetable}{\theenumi}
\setlength{\intextsep}{10pt}

\numberwithin{equation}{enumi}
\numberwithin{figure}{enumi}

\textbf{Question}
Find the roots of quadratic equation:\\
\begin{align}
    x^2 - 3x - 10 = 0
\end{align}

\section*{Solution}
The given equation can be solved using analytical and numerical methods. Let us explore both approaches.

\subsection*{ Quadratic Formula}
The standard quadratic equation is:
\begin{align}
    ax^2 + bx + c = 0
\end{align}
Here, \( a = 1, b = -3, c = -10 \). The roots are given by:
\begin{align}
    x = \frac{-b \pm \sqrt{b^2 - 4ac}}{2a}
\end{align}
Substitute the values of \( a, b, \) and \( c \):
\begin{align}
    x = \frac{-(-3) \pm \sqrt{(-3)^2 - 4(1)(-10)}}{2(1)} \\
    x = \frac{3 \pm \sqrt{9 + 40}}{2} \\
    x = \frac{3 \pm \sqrt{49}}{2}
\end{align}
Simplify further:
\begin{align}
    x_1 = \frac{3 + 7}{2} = 5, \quad x_2 = \frac{3 - 7}{2} = -2
\end{align}

Thus, the roots of the equation are:
\begin{align}
    x_1 = 5, \quad x_2 = -2
\end{align}

\subsection*{ Solution using Fixed Point Iteration}
We rewrite the equation as:
\begin{align}
    x = g(x)
\end{align}
A possible choice for \( g(x) \) is:
\begin{align}
    g(x) = \sqrt{3x + 10}
\end{align}
The iterative update becomes:
\begin{align}
    x_{n+1} = \sqrt{3x_n + 10}
\end{align}

Starting with an initial guess \( x_0 = 2 \), the iterations are as follows:
\begin{align}
    x_1 = \sqrt{3(2) + 10} = \sqrt{16} = 4 \\
    x_2 = \sqrt{3(4) + 10} = \sqrt{22} \approx 4.69 \\
    x_3 = \sqrt{3(4.69) + 10} \approx 5.14 \\
    \vdots
\end{align}
\textbf{Observation:} The iterations converge to \( x = 5 \), one of the roots of the equation. For \( x_2 = -2 \), a similar setup with \( g(x) = -\sqrt{3x + 10} \) would be used.

\subsection*{ Newton-Raphson Method}
The Newton-Raphson method is defined as:
\begin{align}
    x_{n+1} = x_n - \frac{f(x_n)}{f'(x_n)}
\end{align}
Here:
\begin{align}
    f(x) = x^2 - 3x - 10, \quad f'(x) = 2x - 3
\end{align}
Substitute into the formula:
\begin{align}
    x_{n+1} = x_n - \frac{x_n^2 - 3x_n - 10}{2x_n - 3}
\end{align}

\textbf{Example: Starting with an initial guess \( x_0 = 3 \):}
\begin{align}
    x_1 = 3 - \frac{3^2 - 3(3) - 10}{2(3) - 3} = 3 - \frac{9 - 9 - 10}{6 - 3} = 3 + \frac{10}{3} \approx 6.33 \\
    x_2 = 6.33 - \frac{6.33^2 - 3(6.33) - 10}{2(6.33) - 3} \approx 5.02
\end{align}
\textbf{Observation:} The iterations quickly converge to \( x = 5 \). Similarly, starting with \( x_0 = -1 \) converges to \( x = -2 \).

\section*{Computational Approach}
The following results were obtained using a computational method:
\begin{align*}
\text{Running Fixed Point Iterations Method:} 
\text{Root 1:} & \; 5 \\
\text{Root 2:} & \; -2 \\
\text{Running Newton-Raphson Method:} & \\
\text{Root 1:} & \; 5 \\
\text{Root 2:} & \; -2
\end{align*}


\section*{Conclusion}
The roots of the quadratic equation \( x^2 - 3x - 10 = 0 \) are:
\begin{align}
    x_1 = 5, \quad x_2 = -2
\end{align}
Both numerical and analytical methods confirm these results.
\end{document}
