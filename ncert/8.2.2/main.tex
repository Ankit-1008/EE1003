\documentclass[journal]{IEEEtran}
\usepackage[a5paper, margin=10mm, onecolumn]{geometry}
\usepackage{lmodern}
\usepackage{amsmath, amssymb}
\usepackage{graphicx}
\usepackage{hyperref}
\usepackage{bm} 

\title{8.2.2}
\author{EE24BTECH11004 - Ankit Jainar}
\date{}

\begin{document}

\bibliographystyle{IEEEtran}
\vspace{3cm}

\maketitle

\bigskip

\textbf{Question:}
Find the area bounded by the curves:
\begin{align}
    (x - 1)^2 + y^2 = 1 \quad \text{and} \quad x^2 + y^2 = 1
\end{align}
\newline

\textbf{Theoretical Solution:}

The curves are two circles:
\begin{itemize}
    \item Circle 1: $(x - 1)^2 + y^2 = 1$, centered at $(1, 0)$ with radius $1$.
    \item Circle 2: $x^2 + y^2 = 1$, centered at $(0, 0)$ with radius $1$.
\end{itemize}

\textbf{Step 1: Finding the Points of Intersection Using Theoritical Approach}

The points of intersection occur where the two circles overlap. To find the intersection points, we rewrite the equations as:
\begin{align}
    (x - 1)^2 + y^2 &= 1 \quad \text{(Circle 1)} \\
    x^2 + y^2 &= 1 \quad \text{(Circle 2)}
\end{align}

Expanding both equations:
\begin{align}
    x^2 - 2x + 1 + y^2 &= 1 \quad \text{(Circle 1)} \\
    x^2 + y^2 &= 1 \quad \text{(Circle 2)}
\end{align}

Subtracting the second equation from the first gives:
\begin{align}
    -2x + 1 &= 0 \quad \implies \quad x = \frac{1}{2}
\end{align}

Substituting $x = \frac{1}{2}$ into the equation of Circle 2:
\begin{align}
    \left(\frac{1}{2}\right)^2 + y^2 &= 1 \\
    \frac{1}{4} + y^2 &= 1 \\
    y^2 &= \frac{3}{4} \quad \implies \quad y = \pm \frac{\sqrt{3}}{2}
\end{align}

Thus, the points of intersection are:
\begin{align}
    P_1 = \left(\frac{1}{2}, \frac{\sqrt{3}}{2}\right), \quad P_2 = \left(\frac{1}{2}, -\frac{\sqrt{3}}{2}\right)
\end{align}

Matrix representation for solving this system:
\begin{align}
    \bm{A} \cdot \bm{x} = \bm{b},
\end{align}
where:
\begin{align}
    \bm{A} = \begin{bmatrix}
        2 & 0 \\
        1 & 1
    \end{bmatrix}, \quad
    \bm{x} = \begin{bmatrix}
        x \\ y^2
    \end{bmatrix}, \quad
    \bm{b} = \begin{bmatrix}
        1 \\ 1
    \end{bmatrix}.
\end{align}

\textbf{Step 2:Intersection of Two Circles Using Matrix Approach}

\textbf{ General Form of the Circle Equations}

The general form for a circle is:
\begin{align}
    g(x, y) &= \vec{x}^\top V \vec{x} + 2 \vec{u}^\top \vec{x} + f = 0
\end{align}

For Circle 1: \( (x - 1)^2 + y^2 = 1 \)
\begin{align}
    g_1(x, y) &= \vec{x}^\top V_1 \vec{x} + 2 \vec{u}_1^\top \vec{x} + f_1 = 0 \\
    V_1 &= \begin{pmatrix} 1 & 0 \\ 0 & 1 \end{pmatrix}, \quad
    \vec{u}_1 = \begin{pmatrix} -1 \\ 0 \end{pmatrix}, \quad
    f_1 = 0
\end{align}

For Circle 2: \( x^2 + y^2 = 1 \)
\begin{align}
    g_2(x, y) &= \vec{x}^\top V_2 \vec{x} + 2 \vec{u}_2^\top \vec{x} + f_2 = 0 \\
    V_2 &= \begin{pmatrix} 1 & 0 \\ 0 & 1 \end{pmatrix}, \quad
    \vec{u}_2 = \begin{pmatrix} 0 \\ 0 \end{pmatrix}, \quad
    f_2 = -1
\end{align}

\textbf{ Subtract the Equations to Eliminate Quadratic Terms}

Subtract \( g_1(x, y) \) from \( g_2(x, y) \):
\begin{align}
    g_2(x, y) - g_1(x, y) &= (\vec{x}^\top V_2 \vec{x} + 2 \vec{u}_2^\top \vec{x} + f_2) - (\vec{x}^\top V_1 \vec{x} + 2 \vec{u}_1^\top \vec{x} + f_1) = 0 \\
    0 &= 2 (\vec{u}_2 - \vec{u}_1)^\top \vec{x} + (f_2 - f_1)
\end{align}

Simplify:
\begin{align}
    2 \begin{pmatrix} 0 - (-1) \\ 0 - 0 \end{pmatrix}^\top \vec{x} + (-1 - 0) = 0 \\
    2 \begin{pmatrix} 1 \\ 0 \end{pmatrix}^\top \vec{x} - 1 = 0
\end{align}

\textbf{ Line Equation Representing the Chord of Intersection}

The line equation representing the chord of intersection is:
\begin{align}
    \begin{pmatrix} 1 \\ 0 \end{pmatrix}^\top \vec{x} = \frac{1}{2}
\end{align}

This simplifies to:
\begin{align}
    x = \frac{1}{2}.
\end{align}

\textbf{ Substituting \( x \) into One Circle Equation}

Substitute \( x = \frac{1}{2} \) into the second circle equation \( g_2(x, y) \):
\begin{align}
    \left( \frac{1}{2} \right)^2 + y^2 &= 1 \\
    \frac{1}{4} + y^2 &= 1 \\
    y^2 &= \frac{3}{4}.
\end{align}

Solve for \( y \):
\begin{align}
    y = \pm \frac{\sqrt{3}}{2}.
\end{align}

\textbf{ Points of Intersection}

The points of intersection are:
\begin{align}
    \vec{x}_1 &= \begin{pmatrix} \frac{1}{2} \\ \frac{\sqrt{3}}{2} \end{pmatrix}, \quad
    \vec{x}_2 = \begin{pmatrix} \frac{1}{2} \\ -\frac{\sqrt{3}}{2} \end{pmatrix}.
\end{align}


\textbf{Step 3: Area Calculation Using Trapezoidal Rule}

The area between the two circles is symmetric about the $x$-axis. Therefore, we calculate the area of the upper region and multiply it by $2$.


The area is given by:
\begin{align}
    A = 2 \int_{x=0.5}^{x=1} \left[ \sqrt{1 - (x - 1)^2} - \sqrt{1 - x^2} \right] dx
\end{align}

Expanding $1 - (x - 1)^2$:
\begin{align}
    1 - (x - 1)^2 &= 1 - (x^2 - 2x + 1) \\
    &= 2x - x^2
\end{align}

Thus, the integral becomes:
\begin{align}
    A = 2 \int_{x=0.5}^{x=1} \left[ \sqrt{2x - x^2} - \sqrt{1 - x^2} \right] dx
\end{align}

\textbf{Trapezoidal Rule:}

We discretize the interval $[0.5, 1]$ into $N$ equal subintervals of width $h$:
\begin{align}
    h = \frac{1 - 0.5}{N} = \frac{0.5}{N}.
\end{align}

The $x_k$ values are:
\begin{align}
    x_k = 0.5 + k \cdot h, \quad k = 0, 1, 2, \dots, N.
\end{align}

The area is approximated as:
\begin{align}
    A \approx 2 \cdot h \cdot \left[ \frac{1}{2}(f(x_0) - g(x_0)) + \sum_{k=1}^{N-1} (f(x_k) - g(x_k)) + \frac{1}{2}(f(x_N) - g(x_N)) \right],
\end{align}
where:
\begin{align}
    f(x_k) = \sqrt{2x_k - x_k^2}, \quad g(x_k) = \sqrt{1 - x_k^2}.
\end{align}

\textbf{Resultant Equation:}

The iterative formula for the trapezoidal rule is:
\begin{align}
    A_{\text{new}} = A_{\text{old}} + h \cdot \left[(f(x_{k+1}) - g(x_{k+1})) + (f(x_k) - g(x_k)) \right].
\end{align}

\textbf{Step 4: Numerical Integration:}

By substituting the values of $x_k$, $f(x_k)$, and $g(x_k)$ into the trapezoidal rule, we compute the area iteratively.


\begin{figure}[h!]
    \centering
    \includegraphics[width=\columnwidth]{figs/fig.png} 
\end{figure}

\end{document}

