\documentclass[journal]{IEEEtran}
\usepackage[a5paper, margin=10mm, onecolumn]{geometry}
\usepackage{tfrupee}
\usepackage{cite}
\usepackage{amsmath,amssymb,amsfonts,amsthm}
\usepackage{algorithmic}
\usepackage{graphicx}
\usepackage{textcomp}
\usepackage{xcolor}
\usepackage{listings}
\usepackage{enumitem}
\usepackage{mathtools}
\usepackage{gensymb}
\usepackage{comment}
\usepackage[breaklinks=true]{hyperref}
\usepackage{tkz-euclide}
\usepackage{longtable}
\usepackage{multirow}
\usepackage{hhline}
\usepackage{array}

\newcommand{\myvec}[1]{\begin{bmatrix}#1\end{bmatrix}}

\begin{document}
\title{3.2.1.2}
\author{EE24BTECH11004 - Ankit Jainar}
\maketitle

\textbf{Question:} \(5\) pencils and \(7\) pens together cost Rs.\(50\), whereas \(7\) pencils and \(5\) pens together cost Rs.\(46\). Find the cost of one pencil and that of one pen.\\

\section*{Solution:}
Let the cost of one pencil be denoted by \(x\) and the cost of one pen by \(y\).  
The situation can be described using the following system of linear equations:
\begin{align}
    5x + 7y &= 50, \tag{1} \\
    7x + 5y &= 46. \tag{2}
\end{align}

\section{Theoretical Solution}
We solve the above equations using elimination:
\begin{itemize}
    \item Multiply equation \((1)\) by \(5\) and equation \((2)\) by \(7\).
    \item Subtract the resulting equations to eliminate \(y\) and solve for \(x\).
    \item Substitute the value of \(x\) back into either equation to find \(y\).
\end{itemize}

Performing these steps:
\[
x = 3, \quad y = 5.
\]
\section{Numerical Method:}
\section{LU Decomposition to Solve the System}
We now solve the system of equations using LU decomposition.

\subsection{Matrix Form}
The system of equations can be expressed in matrix form as:
\begin{equation}
    \myvec{
    5 & 7 \\
    7 & 5
    } \myvec{x \\ y} = \myvec{50 \\ 46}.
\end{equation}
Here, the coefficient matrix is:
\begin{equation}
    A = \myvec
    {5 & 7 \\
    7 & 5}, \quad \vec{b} = \myvec{50 \\ 46}.
\end{equation}

\subsection{Step 1: Decomposing \(A\) into \(L\) and \(U\)}
The matrix \(A\) can be decomposed into:
\begin{equation}
    A = L \cdot U,
\end{equation}
where:
\begin{align}
    L &= \myvec{1 & 0 \\ \frac{7}{5} & 1}, \\
    U &= \myvec{5 & 7 \\ 0 & -\frac{14}{5}}.
\end{align}

\subsection{Step 2: Forward Substitution}
The system \(A\vec{x} = \vec{b}\) is transformed into \(L \cdot U \cdot \vec{x} = \vec{b}\). Let \(\vec{y}\) satisfy \(L\vec{y} = \vec{b}\):
\begin{equation}
    \myvec{1 & 0 \\ \frac{7}{5} & 1} \myvec{y_1 \\ y_2} = \myvec{50 \\ 46}.
\end{equation}

Using forward substitution:
\begin{align}
    y_1 &= 50, \\
    \frac{7}{5}y_1 + y_2 &= 46 \implies y_2 = 46 - \frac{7}{5}(50) = -24.
\end{align}
Thus:
\begin{equation}
    \vec{y} = \myvec{50 \\ -24}.
\end{equation}

\subsection{Step 3: Back Substitution}
Next, solve \(U\vec{x} = \vec{y}\):
\begin{equation}
    \myvec{5 & 7 \\ 0 & -\frac{14}{5}} \myvec{x \\ y} = \myvec{50 \\ -24}.
\end{equation}

Using back substitution:
\begin{align}
    -\frac{14}{5}y &= -24 \implies y = 5, \\
    5x + 7(5) &= 50 \implies x = 3.
\end{align}
\subsection{Updated Equation:}
\begin{align}
    A\vec{x} &= L \cdot U \cdot \vec{x} = \vec{b}, \\
    A &= L \cdot U, \\
    L \cdot U \cdot \vec{x} &= \vec{b}, \\
    U \cdot \vec{x} &= \vec{y}, \\
    L \cdot \vec{y} &= \vec{b}.
\end{align}
\subsection{Final Answer}
The cost of one pencil is Rs.\(3\), and the cost of one pen is Rs.\(5\).

\begin{figure}[h!]
   \centering
   \includegraphics[width=0.7\columnwidth]{figs/fig.png}
    
\end{figure}

\end{document}

