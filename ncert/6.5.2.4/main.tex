\documentclass[journal]{IEEEtran}
\usepackage[a5paper, margin=10mm, onecolumn]{geometry}
\usepackage{tfrupee}

\setlength{\headheight}{1cm}
\setlength{\headsep}{0mm}

\usepackage{gvv-book}
\usepackage{gvv}
\usepackage{cite}
\usepackage{amsmath,amssymb,amsfonts,amsthm}
\usepackage{algorithmic}
\usepackage{graphicx}
\usepackage{textcomp}
\usepackage{xcolor}
\usepackage{txfonts}
\usepackage{listings}
\usepackage{enumitem}
\usepackage{mathtools}
\usepackage{gensymb}
\usepackage{comment}
\usepackage[breaklinks=true]{hyperref}
\usepackage{tkz-euclide}
\usepackage{listings}
\def\inputGnumericTable{}
\usepackage[latin1]{inputenc}
\usepackage{color}
\usepackage{array}
\usepackage{longtable}
\usepackage{calc}
\usepackage{multirow}
\usepackage{hhline}
\usepackage{ifthen}
\usepackage{lscape}

\begin{document}

\bibliographystyle{IEEEtran}
\vspace{3cm}

\title{6.5.2.4}
\author{EE24BTECH11004 - Ankit Jainar}
\maketitle

\renewcommand{\thefigure}{\theenumi}
\renewcommand{\thetable}{\theenumi}
\setlength{\intextsep}{10pt}

\numberwithin{equation}{enumi}
\numberwithin{figure}{enumi}
\renewcommand{\thetable}{\theenumi}

\textbf{Question}:\\
Find the local minimum/maximum of the given function:\\
$f(x) = |\sin(4x) + 3|$
\\

\section*{Theoretical Method}
We analyze the function theoretically to find its critical points. Let:
\begin{align}
    f(x) = |g(x)|, \quad g(x) = \sin(4x) + 3
\end{align}

The critical points of \(g(x)\) occur where \(g'(x) = 0\). Differentiating \(g(x)\):
\begin{align}
    g'(x) = 4\cos(4x)
\end{align}
Setting \(g'(x) = 0\), we find:
\begin{align}
    \cos(4x) = 0 \implies 4x = \frac{\pi}{2} + n\pi \implies x = \frac{\pi}{8} + \frac{n\pi}{4}, \; n \in \mathbb{Z}
\end{align}

For these \(x\)-values, we calculate \(g(x)\) to find the maximum and minimum values of \(|g(x)|\):
\begin{align}
    g(x) = \sin(4x) + 3, \quad f(x) = |g(x)|
\end{align}
At the critical points, evaluate \(f(x)\) directly to determine the local maximum and minimum values. The function \(f(x)\) achieves its minimum value at \(f(x) = 3\) and maximum value at \(f(x) = 4\).

\section*{Computational Method}
We use gradient descent to find the local minima and maxima numerically. Gradient descent works iteratively using the following update rule:
\begin{align}
    x_{n+1} &= x_n - \mu f^{\prime}(x_n)
\end{align}

The gradient \(f^{\prime}(x)\) is computed numerically as:
\begin{align}
    f^{\prime}(x) \approx \frac{f(x + \delta) - f(x - \delta)}{2\delta}
\end{align}

\noindent \textbf{Algorithm Steps:}
\begin{itemize}
    \item \textbf{Initialize:} Start with an initial guess for \(x\), a step size (\(\mu\)), and a threshold for convergence.
    \item \textbf{Update:} Compute \(f^{\prime}(x_n)\) and update \(x_n\) using the update rule.
    \item \textbf{Convergence:} Stop when \(|f^{\prime}(x_n)| < \text{threshold}\).
\end{itemize}

\noindent \textbf{Results:}
Using an initial guess \(x = 0\), step size \(\mu = 0.01\), and threshold \(1e-5\), the numerical method yields:
\begin{align}
    x_{min} &= 0.785398, \; f(x_{min}) = 3.000000\\
    x_{max} &= 0.392699, \; f(x_{max}) = 4.000000
\end{align}

Thus, the maximum value of \(f(x)\) is 4, and the minimum value of \(f(x)\) is 3. These values match the theoretical results.

\begin{figure}[h!]
   \centering
   \includegraphics[width=0.7\columnwidth]{figs/fig.png}
\end{figure}

\end{document}

